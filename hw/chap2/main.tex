\documentclass{scrartcl}
\usepackage[margin=3cm]{geometry}
\usepackage{amsmath}
\usepackage{amssymb}
\usepackage{amsthm}
\usepackage{blindtext}
\usepackage{datetime}
\usepackage{fontspec}
\usepackage{float}
\usepackage{graphicx}
\usepackage{kotex}
\usepackage[lighttt]{lmodern}
\usepackage{listings}
\usepackage{mathrsfs}
\usepackage{mathtools}
\usepackage{pgf,tikz,pgfplots}

\pgfplotsset{compat=1.15}
\usetikzlibrary{arrows}
\newtheorem{theorem}{Theorem}

\lstset{
  numbers=none, frame=single, showspaces=false,
  showstringspaces=false, showtabs=false, breaklines=true, showlines=true,
  breakatwhitespace=true, basicstyle=\ttfamily, keywordstyle=\bfseries, basewidth=0.5em
}

\setmainhangulfont{Noto Serif CJK KR}[
  UprightFont=* Light, BoldFont=* Bold,
  Script=Hangul, Language=Korean, AutoFakeSlant,
]
\setsanshangulfont{Noto Sans CJK KR}[
  UprightFont=* DemiLight, BoldFont=* Medium,
  Script=Hangul, Language=Korean
]
\setmathhangulfont{Noto Sans CJK KR}[
  SizeFeatures={
    {Size=-6,  Font=* Medium},
    {Size=6-9, Font=*},
    {Size=9-,  Font=* DemiLight},
  },
  Script=Hangul, Language=Korean
]
\title{CSED353: Chap. 2 Exercises (due Mar. 18)}
\author{손량(20220323)}
\date{Last compiled on: \today, \currenttime}

\newcommand{\un}[1]{\ensuremath{\ \mathrm{#1}}}

\begin{document}
\maketitle

\section{Problem \#1}

\subsection{Solution for (a)}
From the parameters given in the question, we can determine the parameters for
internet access link as follows:
\begin{align*}
  L = 2,000,000 \un{bits/req} \quad
  R = 54,000,000 \un{bps} \quad
  a = 20 \un{req/s}
\end{align*}
Then, the utilization for access link is
\begin{align*}
  \rho = \frac{La}{R} = 0.741
\end{align*}
So the access link delay can be calculated as
\begin{align*}
  \frac{L}{R} \left( \frac{\rho}{1 - \rho} \right) + \frac{L}{R}
  = 0.143 \un{s}
\end{align*}
For LAN, we can use \(R = 10,000,000,000 \un{bps}\) and obtain the LAN
utilization of
\begin{align*}
  \rho = \frac{La}{R} = 0.004
\end{align*}
so the LAN delay can be calculated as
\begin{align*}
  \frac{L}{R} \left( \frac{\rho}{1 - \rho} \right) + \frac{L}{R}
  = 0.000201 \un{s}
\end{align*}
Since the internet delay is given as 3 seconds, we obtain the final result as
\begin{align*}
  0.143 \un{s} + 3 \un{s} + 0.000201 \un{s} = 3.143 \un{s}
\end{align*}

\subsection{Solution for (b)}
By taking cache into account, we can obtain utilization for access link as
\begin{align*} 
  \rho = 0.4 \times \frac{La}{R} = 0.296
\end{align*}
Assuming that round trip from and to the local web cache is negligible in the
case of cache miss, the delay from the origin servers can be obtained as
\begin{align*}
  \frac{L}{R} \left( \frac{\rho}{1 - \rho} \right) + \frac{L}{R} + 3 \un{s}
  = 3.05 \un{s}
\end{align*}
The utilization of LAN is
\begin{align*}
  \rho = 0.6 \times \frac{La}{R} = 0.0240
\end{align*}
Then the delay for cache server can be obtained as
\begin{align*}
  \frac{L}{R} \left( \frac{\rho}{1 - \rho} \right) + \frac{L}{R}
  = 0.00205 \un{s}
\end{align*}
The final result is
\begin{align*}
  0.4 \times 3.05 \un{s} + 0.6 \times 0.00205 \un{s} = 1.22 \un{s}
\end{align*}

\end{document}
% vim: textwidth=79
