\documentclass{scrartcl}
\usepackage[margin=3cm]{geometry}
\usepackage{amsmath}
\usepackage{amssymb}
\usepackage{amsthm}
\usepackage{blindtext}
\usepackage{datetime}
\usepackage{fontspec}
\usepackage{float}
\usepackage{graphicx}
\usepackage{kotex}
\usepackage[lighttt]{lmodern}
\usepackage{listings}
\usepackage{mathrsfs}
\usepackage{mathtools}
\usepackage{pgf,tikz,pgfplots}
\usepackage{tabularx}

\pgfplotsset{compat=1.15}
\usetikzlibrary{arrows}
\newtheorem{theorem}{Theorem}

\lstset{
  numbers=none, frame=single, showspaces=false, showstringspaces=false,
  showtabs=false, breaklines=true, showlines=true, breakatwhitespace=true,
  basicstyle=\ttfamily, keywordstyle=\bfseries, basewidth=0.5em
}

\setmainhangulfont{Noto Serif CJK KR}[
  UprightFont=* Light, BoldFont=* Bold,
  Script=Hangul, Language=Korean, AutoFakeSlant,
]
\setsanshangulfont{Noto Sans CJK KR}[
  UprightFont=* DemiLight, BoldFont=* Medium,
  Script=Hangul, Language=Korean
]
\setmathhangulfont{Noto Sans CJK KR}[
  SizeFeatures={
    {Size=-6,  Font=* Medium},
    {Size=6-9, Font=*},
    {Size=9-,  Font=* DemiLight},
  },
  Script=Hangul, Language=Korean
]
\title{CSED353: Chap. 3 Exercises (due Apr. 26)}
\author{손량(20220323)}
\date{Last compiled on: \today, \currenttime}

\newcommand{\un}[1]{\ensuremath{\ \mathrm{#1}}}

\begin{document}
\maketitle

\section{Problem \#1}

\subsection{Solution for (a)}

The hexadecimal representation of the header is \texttt{c5e8c351000c5cf8}, and
\texttt{61626364} for data. We can split the hexadecimal into 16-bit words and
obtain \texttt{[0xc5e8, 0xc351, 0x000c, 0x5cf8, 0x6162, 0x6364]}. The sum can be
calculated using the following code:

\begin{lstlisting}[language=Python]
words = [0xC5E8, 0xC351, 0x000C, 0x5CF8, 0x6162, 0x6364]
checksum = 0
for word in words:
    checksum += word
    carryover = checksum >> 16
    checksum = (checksum & 0xFFFF) + carryover
print(f"checksum: {checksum:04x}")
\end{lstlisting}

and we obtain \texttt{0xab05}.

\subsection{Solution for (b)}

As the checksum is not \texttt{0xffff}, the packet contains errors.

\subsection{Solution for (c)}

The structure of UDP pseudoheader can be represented as follows:

\begin{figure}[H]
\centering
\begin{tabularx}{\textwidth}{ |c| *{4}{X|} }
\hline
Octet & 0 & 1 & 2 & 3 \\ \hline
0 & \multicolumn{4}{l|}{Source IPv4 Address (32 bit)} \\ \hline
4 & \multicolumn{4}{l|}{Destination IPv4 Address (32 bit)} \\ \hline
8 & Zeroes (8 bit) & Protocol (8 bit) & \multicolumn{2}{l|}{UDP Length (16 bit)} \\ \hline
\end{tabularx}
\end{figure}

\subsection{Solution for (d)}

The hexadecimal part for pseudoheader is \texttt{c0a801078ddf054e0011000c},
which can be split as \texttt{[0xc0a8, 0x0107, 0x8ddf, 0x054e, 0x0011, 0x000c]}.
The sum can be calculated using the following code:

\begin{lstlisting}[language=Python]
words = [
    0xC0A8,
    0x0107,
    0x8DDF,
    0x054E,
    0x0011,
    0x000C,
    0xC5E8,
    0xC351,
    0x000C,
    0x5CF8,
    0x6162,
    0x6364,
]
checksum = 0
for word in words:
    checksum += word
    carryover = checksum >> 16
    checksum = (checksum & 0xFFFF) + carryover
print(f"checksum: {checksum:04x}")
\end{lstlisting}

And we obtain \texttt{0xffff}.

\subsection{Solution for (e)}

Taking IPv4 UDP pseudoheader into account, the packet is error-free in
UDP layer as the checksum is valid. However, there might be errors in other
layers, such as application layer so we cannot be entirely sure that packet
is error-free.

\subsection{Solution for (f)}

By using UDP pseudoheader, the UDP checksum mechanism can detect errors outside
UDP layer, such as incorrect source/destination address and routing errors.

\section{Problem \#2}

Let \(L = 1500 \un{bytes}, R = 10^9 \un{bits/s}, \mathsf{RTT} = 30 \un{ms}\) and
\(n\) as window size, in number of packets. Then the utilization condition can
be written as

\begin{align*}
\frac{nL / R}{\mathsf{RTT} + nL / R} > 0.98
\end{align*}

Solving this inequality, we obtain \(n > 122500\)\footnote{Of course, we only
consider positive values of \(n\).} so the window size should be bigger than
122500 bits to have utilization greater than 98 percent.

\section{Problem \#3}

\subsection{Solution for (a)}

As written in the problem statement, the initial sequence number at \(t = 1\)
is 111.

\subsection{Solution for (b)}

Since the segment sent at \(t = 1\) is the initial segment, the segment has
SYN flag set and contains payload of size 926 bytes. So the segment's length in
sequence space is 907. Thus, the sequence number of the segment sent at \(t =
2\) is \(111 + 927 = 1038\).

\subsection{Solution for (c)}

The length of the segment sent in \(t = 2\), in sequence space is 926, as both
SYN and FIN flag are not set. Thus, the sequence number of the segment sent in
\(t = 3\) is \(111 + 927 + 926 = 1964\).

\subsection{Solution for (d)}

The length of the segment sent in \(t = 3\), in sequence space is 926, like the
segment sent in \(t = 2\). Thus, the sequence number of the segment sent in \(t
= 4\) is \(111 + 927 + 926 + 926 = 2890\).

\subsection{Solution for (e)}

The length of the segment sent in \(t = 4\), in sequence space is 926, like the
segment sent in \(t = 3\). Thus, the sequence number of the segment sent in \(t
= 5\) is \(111 + 927 + 926 + 926 + 926 = 3816\).

\subsection{Solution for (f)}

Only segment sent in \(t = 1\) has arrived in \(t = 8\), so ACK number is \(111
+ 927 = 1038\).

\subsection{Solution for (g)}

Segment sent in \(t = 1\) and \(t = 2\) have arrived in \(t = 9\), so ACK number
is \(111 + 927 + 926 = 1964\).

\subsection{Solution for (h)}

The segment is lost.

\subsection{Solution for (i)}

Since the segment sent in \(t = 2\) is the last in-order segment in \(t = 11\),
the ACK number is \(111 + 927 + 926 = 1964\).

\subsection{Solution for (j)}

The segment is lost.

\subsection{Solution for (k)}

The segment sent in \(t = 1\) is acknowledged, so the sixth segment is sent in
order to fill the window, whose sequence number is \(111 + 927 + 926 + 926 + 926
+ 926 = 4742\).

\subsection{Solution for (l)}

The segment sent in \(t = 2\) is yet to be acknowledged, so the sender does not
send any segment. None.

\subsection{Solution for (m)}

The segment sent in \(t = 2\) is yet to be acknowledged, so the sender does not
send any segment. None.

\subsection{Solution for (n)}

The segment sent in \(t = 2\) is acknowledged, so the seventh segment is sent in
order to fill the window, whose sequence number is \(111 + 927 + 926 + 926 + 926
+ 926 + 926 = 5668\).

\subsection{Solution for (o)}

The segment sent in \(t = 3\) is yet to be acknowledged, so the sender does not
send any segment. None.

\section{Problem \#4}

For convenience, we denote values of \(\mathsf{EstimatedRTT}\),
\(\mathsf{DevRTT}\) and \(\mathsf{TimeoutInterval}\) after \(n\)-th
RTT as \(\mathsf{EstimatedRTT}(n)\), \(\mathsf{DevRTT}(n)\) and
\(\mathsf{TimeoutInterval}\), respectively. Naturally, \(\mathsf{EstimatedRTT}
(0) = 320\) and \(\mathsf{DevRTT}(0) = 39\). In addition, let \(n\)-th measured
RTT as \(\mathsf{SampleRTT}(n)\).

\subsection{Solution for (a)}

We can calculate
\begin{align*}
\mathsf{EstimatedRTT}(1)
&= (1 - \alpha) \times \mathsf{EstimatedRTT}(0) + \alpha \times
\mathsf{SampleRTT}(1) \\
&= (1 - 0.125) \times 320 + 0.125 \times 390
= 328.75
\end{align*}

\subsection{Solution for (b)}

We can calculate
\begin{align*}
\mathsf{DevRTT}(1)
&= (1 - \beta) \times \mathsf{DevRTT}(0) + \beta \times |\mathsf{SampleRTT}(1) -
\mathsf{EstimatedRTT}(0)| \\
&= (1 - 0.25) \times 39 + 0.25 \times |390 - 320|
= 46.75
\end{align*}

\subsection{Solution for (c)}

We can calculate
\begin{align*}
\mathsf{TimeoutInterval}(1)
= \mathsf{EstimatedRTT}(1) + 4 \times \mathsf{DevRTT}(1)
= 515.75
\end{align*}

\subsection{Solution for (d)}

We can calculate
\begin{align*}
\mathsf{EstimatedRTT}(2)
&= (1 - \alpha) \times \mathsf{EstimatedRTT}(1) + \alpha \times
\mathsf{SampleRTT}(2) \\
&= (1 - 0.125) \times 328.75 + 0.125 \times 270
= 321.41
\end{align*}

\subsection{Solution for (e)}

We can calculate
\begin{align*}
\mathsf{DevRTT}(2)
&= (1 - \beta) \times \mathsf{DevRTT}(1) + \beta \times |\mathsf{SampleRTT}(2) -
\mathsf{EstimatedRTT}(1)| \\
&= (1 - 0.25) \times 46.75 + 0.25 \times |270 - 328.75|
= 49.75
\end{align*}

\subsection{Solution for (f)}

We can calculate
\begin{align*}
\mathsf{TimeoutInterval}(2)
= \mathsf{EstimatedRTT}(2) + 4 \times \mathsf{DevRTT}(2)
= 520.41
\end{align*}

\section{Problem \#5}

\subsection{Solution for (a)}

Since the server does not keep track of SYN segments, the server has to store
the information elsewhere. SYN cookie utilizes the sequence number as the
alternative storage.

\subsection{Solution for (b)}

The attacker has to guess the server's initial sequence number, and that
involves guessing the server's secret number which is hashed alongside other
data. The attacker won't be able to create connection as long as the server's
secret number stays secret.

\subsection{Solution for (c)}

If all the data that is hashed alongside the secret value can be controlled by
the attacker, it is possible for the attacker to create fully open connections.
Since the server does not keep track of informations about SYN, the attacker
can simply send ACK segment with appropriate fields (which depends on how
SYN cookies are constructed in the server), and with sequence number of
\(\mathsf{CollectedISN + 1}\) and perform SYN flooding attack. The server won't
be able to determine whether those ACK segments are legitimate, as they have no
information about SYN segments received in the past.

\end{document}
% vim: textwidth=79
